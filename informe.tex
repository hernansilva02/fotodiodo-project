\documentclass[a4paper,conference]{IEEEtran}
\usepackage{graphicx}
\usepackage{amsmath}
\usepackage{url}

\renewcommand{\abstractname}{Abstracto}

\makeatletter
\newcommand{\authornewline}{
        \end{@IEEEauthorhalign}
        \hfill\mbox{}\par\mbox{}\hfill
        \begin{@IEEEauthorhalign}
}
\makeatother

\author{
    \IEEEauthorblockN{Hernán Alejandro Silva}
    \IEEEauthorblockA{
        Facultad Regional Avellaneda\\
        Universidad Tecnológica Nacional\\
        Buenos Aires, Argentina\\
        hernansilva2002@gmail.com
    }
    \and
    \IEEEauthorblockN{Elías Ramírez}
    \IEEEauthorblockA{
        Facultad Regional Avellaneda\\
        Universidad Tecnológica Nacional\\
        Buenos Aires, Argentina\\
        foo@gmail.com
    }
    \and
    \IEEEauthorblockN{Florencia Mincone}
    \IEEEauthorblockA{
        Facultad Regional Avellaneda\\
        Universidad Tecnológica Nacional\\
        Buenos Aires, Argentina\\
        foo@gmail.com
    }
    \authornewline
    \IEEEauthorblockN{Nicolás Lahorca}
    \IEEEauthorblockA{
        Facultad Regional Avellaneda\\
        Universidad Tecnológica Nacional\\
        Buenos Aires, Argentina\\
        foo@gmail.com
    }
    \and
    \IEEEauthorblockN{Luciano Justiniano}
    \IEEEauthorblockA{
        Facultad Regional Avellaneda\\
        Universidad Tecnológica Nacional\\
        Buenos Aires, Argentina\\
        foo@gmail.com
    }
}
\title{Detector de Partículas Beta}

\begin{document}
\maketitle
\begin{abstract}
    Constantemente los objetos que nos rodean emiten partículas que los
    sentidos humanos no son capaces de percibir. Estas partículas pueden ser
    perjudiciales para la salud y es necesario cuantificarlas para evitar o
    reducir la exposición a ellas. Para lograr ese objetivo, en este documento
    se presenta la realización de un dispositivo que cumpla la función de
    cuantificar un tipo de radiación, llamada radiación $\boldsymbol{\beta}$. En
    particular, se hará hincapié en la radiación por emisión de electrones, a
    este tipo de radiación se la conoce como radiación $\boldsymbol{\beta-}$.
    Además, se propone el análisis de su principio de funcionamiento, los
    materiales necesarios para su construcción y sus limitaciones.
\end{abstract}
\section{Introducción}
    El presente documento sirve como informe sobre el proyecto de fin de año de
    la asignatura Física Electrónica. Dicho proyecto se trata de un detector y
    contador de partículas beta, cubriendo de esta forma el tema de "Radiación"
    de la asignatura. Para más información sobre el proyecto, se recomienda
    visitar el repositorio del mismo que se encuentra en el siguiente enlace:
    (enlace al repo del proyecto). %TODO: crear repo en github y adjunarlo aca
\section{Principio de Funcionamiento}
    \subsection{¿Que son las radiaciones $\beta$?}
        Las radiaciones $\beta$ son un tipo de radiacion ionizante, la cúal se
        caracteriza por emitirse durante el proceso de desintegración o
        decaimento $\beta$. Dicho proceso puede producirse de dos diferentes
        formas.
        \begin{enumerate} 
            \item \textit{Emisión de electrones}: Un núcleo inestable de un
                átomo emite un electrón y un antineutrino, conviritiendo de esta
                manera un neutrón en un protón. A este proceso se lo conoce como
                como "\textbf{Desintegración $\boldsymbol{\beta-}$}". 
            \item \textit{Emisión de positrones}: El núcleo inestable emite un
                positrón, es decir, un electrón cargado positivamente, junto con
                un neutrino. De esta forma se logra transformar un protón en un
                neutrón. Este proceso recibe el nombre de
                "\textbf{Desintegración $\boldsymbol{\beta+}$}". 
        \end{enumerate}

    \subsection{Formas de captar las partículas}

        

% Las malditas referencias. TODO: Sacar este comment.
\bibliographystyle{IEEEtran}
\bibliography{refs}
\nocite{*}
\end{document}
