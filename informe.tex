\documentclass[a4paper,11pt,conference]{IEEEtran}
\usepackage{graphicx}
\usepackage{amsmath}
\usepackage{url}

\renewcommand{\abstractname}{Abstracto}

\author{
    \IEEEauthorblockN{Hernán Alejandro Silva}
    \IEEEauthorblockA{
        Facultad Regional Avellaneda\\
        Universidad Tecnológica Nacional\\
        Buenos Aires, Argentina\\
        hernansilva2002@gmail.com
    }
    \and
    \IEEEauthorblockN{Elías Ramírez}
    \IEEEauthorblockA{
        Facultad Regional Avellaneda\\
        Universidad Tecnológica Nacional\\
        Buenos Aires, Argentina\\
        foo@gmail.com
    }
    \and
    \IEEEauthorblockN{Florencia Mincone}
    \IEEEauthorblockA{
        Facultad Regional Avellaneda\\
        Universidad Tecnológica Nacional\\
        Buenos Aires, Argentina\\
        foo@gmail.com
    }
    \and
    \IEEEauthorblockN{Nicolás Lahorca}
    \IEEEauthorblockA{
        Facultad Regional Avellaneda\\
        Universidad Tecnológica Nacional\\
        Buenos Aires, Argentina\\
        foo@gmail.com
    }
    \and
    \IEEEauthorblockN{Luciano Justiniano}
    \IEEEauthorblockA{
        Facultad Regional Avellaneda\\
        Universidad Tecnológica Nacional\\
        Buenos Aires, Argentina\\
        foo@gmail.com
    }
}
\title{Detector de Partículas Beta}

\begin{document}
\maketitle
\begin{abstract}
    Constantemente los objetos que nos rodean emiten radiaciones que los
    sentidos humanos no son capaces de percibir. Estas radiaciones pueden ser
    perjudiciales para la salud y es necesario cuantificarlas para evitar o
    reducir la exposición a ellas. Para lograr ese objetivo, en este documento
    se presenta la realización de un dispositivo que cumpla la función de
    cuantificar un tipo de radiación específica, llamada radiación beta. Se verá
    su principio de funcionamiento, los materiales necesarios para su
    construcción y sus limitaciones.
\end{abstract}
\section{Introducción}
    El presente documento sirve como informe sobre el proyecto de fin de año de la
    asignatura Física Electrónica. Dicho proyecto se trata de un detector y contador
    de partículas beta, cubriendo de esta forma el tema de "Radiación" de la
    asignatura. Para más información sobre el proyecto, se recomienda visitar el
    repositorio del mismo que se encuentra en el siguiente enlace: (enlace al repo
    del proyecto). %TODO: crear repo en github y adjunarlo aca
\section{Principio de Funcionamiento}
    \subsection{¿Que son las radiaciones $\beta$?}
        Las radiaciones $\beta$ son un tipo de radiacion ionizante, en la cuál adsdas
        

% Las malditas referencias. TODO: Sacar esto.
\bibliographystyle{IEEEtran}
\bibliography{refs}
\nocite{*}
\end{document}
